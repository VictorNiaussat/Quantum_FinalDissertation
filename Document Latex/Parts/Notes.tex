Les technologies quantiques espèrent mettre à profit les différences entre la mécanique classique et quantique afin d'exécuter de façon plus rapide, plus efficace ou plus sûre un certain nombre de tâches. C'est ce qu'on appelle l'avantage quantique. Le débat reste ouvert de savoir dans quelle mesure un tel avantage peut être atteint, dans quelles circonstances et pour quel type de tâche. Le but de ce mémoire est de se familiariser avec cette problématique dans le contexte du quantum random sampling et/ou du machine learning quantique. \\

Après s'être familiarisée avec la problématique, l'étudiant adressera quelques-unes des questions mathématiques qui se présentent naturellement dans ce contexte. Le stage vient en complément naturel au cours sur l'information quantique du S4.\\ 

Le sujet du stage est typiquement un sujet de physique mathématique, et se situe à l'intersection de l'analyse fonctionnelle, des probabilités et de la modélisation. Comme plusieurs questions restent ouvertes, des simulations numériques sont intéressantes et peuvent faire partie du stage, selon les goûts de l'étudiant.
